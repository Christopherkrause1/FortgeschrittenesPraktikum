\section{Auswertung}
\label{sec:Auswertung}

\subsection{Impulshöhen}

Die gemessenen Impulssignale mit und ohne Verstärker werden in Abbildung \ref{fig:mit} und \ref{fig:ohne} dargestellt.

\begin{figure}[H]
  \centering
  \includegraphics[height=8cm]{mitverstaerker.png}
  \caption{Impulshöhen mit Verstärker}
  \label{fig:mit}
\end{figure}

\begin{figure}[H]
  \centering
  \includegraphics[height=8cm]{ohneverstaerker.png}
  \caption{Impulshöhen ohne Verstärker}
  \label{fig:ohne}
\end{figure}

Die Anstiegszeit des Signals mit Verstärker beträgt $\SI{1.3}{\micro\second}$ und die ohne Verstärker $\SI{1}{\micro\second}$.




\subsection{Bestimmung der Foliendicke}
In Tabelle 1 sind die gemessenen Spannungen in Abhängigkeit des Kammerdruckes mit und ohne Goldfolie dargestellt.

\begin{table}[H]
\centering
\caption{Spannungen in Abhängigkeit des Kammerdruckes }
\sisetup{table-format=2.1}
\begin{tabular}{S S S| S S S}
  \toprule
    \multicolumn{3}{c}{Ohne Folie} & \multicolumn{3}{c}{Mit Folie} \\
    \cmidrule(lr){1-3}\cmidrule(lr){4-6}
    {$p/$mbar} & {$U/\, \symup{V}$} & {Fehler/V} & {$p$/mbar} & {$U/\, \symup{V}$} & {Fehler/V} \\
    \midrule
    0.09  & 3.66 &  0.5 & 0.12 & 2.66 & 0.5 \\
    10.0   & 3.64 &  0.5 & 10.7 & 2.50 & 0.6 \\
    20.0   & 3.52 &  0.5 & 20.5 & 2.40 & 0.6 \\
    29.8   & 3.32 &  0.5 & 32.3 & 2.30 & 0.6 \\
    40.1   & 3.32 &  0.5 & 40.8 & 2.16 & 0.6 \\
    50.2   & 3.20 &  0.6 & 50.7 & 2.00 & 0.7 \\
    60.1   & 3.12 &  0.6 & 61.2 & 1.88 & 0.6 \\
    70.5   & 3.00 &  0.5 & 72.0 & 1.74 & 0.6 \\
    80.2   & 2.96 &  0.4 & 82.3 & 1.70 & 0.7 \\
    91.8   & 2.88 &  0.5 & 92.2 & 1.58 & 0.7 \\
    100.2  & 2.80 &  0.5 & 100.7& 1.50 & 0.7 \\
    111.2  & 2.70 &  0.4 & 111.0& 1.38 & 0.8 \\
    121.2  & 2.30 &  0.5 & 122.0& 1.16 & 0.5 \\
    132.2  & 2.20 &  0.5 & 131.3& 1.04 & 0.5 \\
    141.7  & 2.10 &  0.5 & 142.9& 0.91 & 0.5 \\
    151.1  & 2.00 &  0.5 & 150.7& 0.76 & 0.3 \\
    162.2  & 1.70 &  0.5 & 160.9& 0.72 & 0.3 \\
    171.2  & 1.64 &  0.5 & 172.7& 0.68 & 0.3 \\
    181.3  & 1.58 &  0.6 & 198.0& 0.40 & 0.2 \\
    190.5  & 1.50 &  0.5 & & & \\
    200.8  & 1.30 &  0.5 & & & \\
    217.6  & 1.00 &  0.6 & & & \\
      \bottomrule
  \end{tabular}
\end{table}

In Abbildung 4 wird  die Spannung in Abhängigkeit des Druckes mit und ohne Goldfolie dargestellt. Dabei wird ein linearer
Zusammenhang zwischen Druck und Spannung angenommen:
\begin{align*}
  U = a \cdot p + b
\end{align*}

\begin{figure}[H]
  \centering
  \includegraphics[height=10cm]{build/pulsmit.pdf}
  \caption{gemessene Spannung in Abhängigkeit des Kammerdruckes.}
  \label{fig:ohne}
\end{figure}

Die Parameter der Funktion mit und ohne Folie betragen:
\begin{align*}
  &a_{\symup{mit}} = \SI{-0.0117(2)}{\volt\per\milli\bar}  \\
  &b_{\symup{mit}} = \SI{2.6(2)}{\volt}  \\
  &a_{\symup{ohne}}= \SI{-0.0123(4)}{\volt\per\milli\bar}  \\
  &b_{\symup{ohne}} = \SI{ 3.82(5)}{\volt} \\
\end{align*}

Unter der Annahme einer linearen Energieabnahme der $\alpha$-Teilchen entspricht $\Delta x$ aus Gleichung (1) der Foliendicke.
Das Verhältnis der Achsenabschnitte $b_{\symup{mit}}$ und $b_{\symup{ohne}}$ entspricht dem Verhältnis $\frac{E_{\symup{mit}}}{E_{\symup{ohne}}}$, wobei
$E_{\symup{mit}}$ und $E_{\symup{ohne}}$ die Energien bei $p=\SI{0}{\milli\bar}$ darstellen.
Für $\Delta x$ aus Gleichung (1) gilt damit:
\begin{align}
  \Delta x = E_{\symup{ohne}} \left(1- \frac{b_{\symup{mit}}}{b_{\symup{ohne}}}\right) \frac{2 m_{\symup{0}} E_{\symup{\alpha}} (4 \pi \epsilon_0)^2}{4 \pi e^2 z_{\symup{\alpha}}  m_{\symup{\alpha}} N_{\symup{Au}} Z_{\symup{Au}} \ln{\left(\frac{4 E_{\symup{\alpha}} m_{\symup{0}}}{m_{\symup{\alpha}} I_{\symup{Au}}}\right)}}
\end{align}

Hierbei gilt für die Werte:
\begin{align*}
  &m_0 = \SI{9.109387e-31}{\kilo\gram} \\
  &m_{\symup{\alpha}} = \SI{6.64465723e-27}{\kilo\gram} \\
  &I_{\symup{Au}} = \SI{709}{\eV} \\
  &E_{\symup{ohne}} = \SI{5.486}{\mega\eV}   \text{\cite{sample3}} \\
  &Z_{\symup{Au}} = 79 \\
  &N_{\symup{Au}} = \frac{Z \cdot \rho_{\symup{Au}}}{A \cdot u} \approx \SI{5.895e22}{\per\centi\meter\squared} \\
  &z_{\symup{\alpha}} = 2 \\
\end{align*}
%\begin{align*}
%  &m_0= \SI{9.1093897e-31}{\kilogram} \\
%  &m_{\symup{\alpha}} = \SI{6.64465723e-27}{\kilogram} \\
%  &I_{\symup{Au}} = \SI{709}{\eV} \cite{sample4}\\
%  &E_{\symup{ohne}} = \SI{5.486}{\mega\eV} \cite{sample3} \\
%  &Z_{\symup{Au}} = 79 \\
%  &N_{\symup{Au}} = \frac{Z \rho_{\symup{Au}}}{A \cdot u} = \SI{5.895e22}{\per\centi\meter\squared} \\
%  &z_{\symup{\alpha}} = 2 \\
%\end{align*}

Mit der Massenzahl $A$, der Dichte von Gold $\rho_{\symup{Au}}= \SI{19.282}{\gram\per\centi\meter\squared}$ und der atomaren Masseneinheit $u$.
Für die Foliendicke ergibt sich:
\begin{align*}
  \Delta x = \SI{3.6(6)}{\micro\meter}
\end{align*}

Der Fehler wird dabei mit der Gaußschen Fehlerfortpflanzung ermittelt:
\begin{align*}
  \sigma_{\symup{x}} = \sqrt{\left(\frac{\partial \Delta x}{\partial b_{\symup{mit}}}\sigma_{\symup{b_{\symup{mit}}}}\right)^2 + \left(\frac{\partial \Delta x}{\partial b_{\symup{ohne}}}\sigma_{\symup{b_{ohne}}}\right)^2}
\end{align*}

\begin{align*}
  \left(\frac{\partial \Delta x}{\partial b_{\symup{mit}}}\sigma_{\symup{b_{mit}}}\right)^2 = \sigma_{b_{\symup{mit}}}^{2} \left(- \frac{c \left(- \frac{b_{\symup{mit}}}{b_{\symup{ohne}}} + 1\right)}{b_{\symup{ohne}} \left(\frac{b_{\symup{mit}}}{b_{\symup{ohne}}} + 1\right) \ln^{2}{\left (d \left(\frac{b_{\symup{mit}}}{b_{\symup{ohne}}} + 1\right) \right )}} - \frac{c}{b_{\symup{ohne}} \ln{\left (d \left(\frac{b_{\symup{mit}}}{b_{\symup{ohne}}} + 1\right) \right )}}\right)^{2} \\
  \left(\frac{\partial \Delta x}{\partial b_{\symup{ohne}}}\sigma_{\symup{b_{ohne}}}\right)^2 = \sigma_{b_{\symup{ohne}}}^{2} \left(\frac{b_{\symup{mit}} c \left(- \frac{b_{\symup{mit}}}{b_{\symup{ohne}}} + 1\right)}{b_{\symup{ohne}}^{2} \left(\frac{b_{\symup{mit}}}{b_{\symup{ohne}}} + 1\right) \ln^{2}{\left (d \left(\frac{b_{\symup{mit}}}{b_{\symup{ohne}}} + 1\right) \right )}} + \frac{b_{\symup{mit}} c}{b_{\symup{ohne}}^{2} \ln{\left (d \left(\frac{b_{\symup{mit}}}{b_{\symup{ohne}}} + 1\right) \right )}}\right)^{2}
\end{align*}
Mit den Werten:
\begin{align*}
  &c =  E_{\symup{ohne}} \frac{2 m_{\symup{0}} E_{\symup{\alpha}} (4 \pi \epsilon_0)^2}{4 \pi e^2 z_{\symup{\alpha}}  m_{\symup{\alpha}} N_{\symup{Au}} Z_{\symup{Au}}} \\
  &d = \frac{2 E_{\symup{ohne}} m_{\symup{0}}}{m_{\symup{\alpha}} I_{\symup{Au}}}
\end{align*}

\subsection{Differentieller Wirkungsquerschnitt}
\label{sec:wq}
In Tabelle 2 wird die Zählrate $N$ in Abhängigkeit des Winkels $\Theta$ und der Messzeit $t$ dargestellt.


\begin{table}[H]
  \centering
  \caption{Zählrate in Abhängigkeit von Winkel und Messzeit}
  \label{tab:Parameter}
  \begin{tabular}{c c c}
    \toprule
    N & $\Theta/$° & $t$ /s\\
    \midrule
    2219 &  0 & 200 \\
    2120 &  1 & 200 \\
    1748 &  2 & 200 \\
    1477 &  3 & 200 \\
    1684 &  4 & 300 \\
    1317 &  5 & 300 \\
    1218 &  6 & 400 \\
    1199 &  7 & 600 \\
     764 &  8 & 600 \\
     519 & 10 & 600 \\
     231 & 12 & 600 \\
      83 & 15 & 600 \\
      34 & 18 & 600 \\
      25 & 20 & 600 \\
    \bottomrule
  \end{tabular}
\end{table}

Der differentielle Wirkungsquerschnitt wird nun aus der Gleichung
\begin{equation}
  \frac{\symup{d}\sigma}{\symup{d}\Omega} = \frac{N}{t \cdot I_0 \cdot N_\symup{Au} \cdot \Delta x \cdot \Delta \Omega}
  \label{eqn:wq}
\end{equation}
berechnet.

Dabei ist $I_0$ die Zählrate der Nullmessung ohne Folie und ergibt sich zu $I_0 =\SI{20.83(26)}{1\per\second}$.

$\Delta \Omega$ bezeichnet den pyramidenartigen Raumwinkel, der von
der effektiven Detektorfläche eingenommen wird. Um diesen zu berechnen, muss
als Erstes die effektive Detektorfläche selbst bestimmt werden. Dies geschieht
mithilfe der Abbildung \ref{fig:aufbau}. Die gesuchte Breite $b$ und Höhe $h$ der
besagten Fläche lassen sich berechnen aus

\begin{align*}
  b &= \frac{b_\symup{b} \cdot d}{c} \\
  h &= \frac{h_\symup{b} \cdot d}{c}
\end{align*}

Dabei bezeichnen $b_\symup{b}$ und $h_\symup{b}$ die Breite und Höhe der Blende.
$c$ ist der Abstand von der Folie zur Blende und $d$ ist der Abstand von der Folie
zur eigentlichen Detektoroberfläche. Es gilt \cite{sample1}:

\begin{align*}
  b_\symup{b} = \SI{2}{\milli\meter} \\
  h_\symup{b} = \SI{10}{\milli\meter} \\
  c = \SI{41}{\milli\meter} \\
  d = \SI{41}{\milli\meter}
\end{align*}
Und daraus ergibt sich
\begin{align*}
  b &= \SI{2.195}{\milli\meter} \\
  h &= \SI{10.976}{\milli\meter}
\end{align*}

Der Raumwinkel ergibt sich dann aus \cite{sample2}
\begin{equation}
  \Delta \Omega = 4 \cdot \arctan\left(\frac{b \cdot h}{2d \cdot \sqrt{4d^2 + b^2 + h^2}}\right) = \SI{0.0118}{}
\end{equation}

$N_\symup{Au}$ ist die Teilchendichte von Gold. Sie berechnet sich mit der Dichte
$\rho_\symup{Au} = \SI{19.32}{\gram\per\centi\meter^3}$ \cite{sample} und molaren Masse $M_\symup{Au} = \SI{196.97}{\gram\per\mol}$ \cite{sample}
von Gold aus
\begin{equation}
  N_\symup{Au} = N_\symup{A} \cdot \frac{\rho_\symup{Au}}{M_\symup{Au}} = 5,907 \cdot 10^{28} \, 1/\symup{m^3}
  \label{eqn:teilchendichte}
\end{equation}


$\Delta x = 2 \cdot 10^{-6} \, \symup{m}$ ist die Dicke der Folie. Daraus können dann die
differentiellen Wirkungsquerschnitte berechnet werden. Die Ergebnisse sind in Tabelle 3
in Abhängigkeit vom Winkel $\Theta$ dargestellt.


\begin{table}[H]
  \centering
  \caption{Berechnete differentielle Wirkungsquerschnitte}
  \label{tab:Parameter}
  \begin{tabular}{c c}
    \toprule
    $\Theta/$° & $\frac{\symup{d}\sigma}{\symup{d}\Omega}/ \mathrm{10^{-22}m^2}$ \\
    \midrule
     0  &  $\SI{3.819(94)}{}$  \\
     1  &  $\SI{3.648(91)}{}$  \\
     2  &  $\SI{3.008(81)}{}$  \\
     3  &  $\SI{2.542(73)}{}$  \\
     4  &  $\SI{1.932(53)}{}$  \\
     5  &  $\SI{1.511(46)}{}$  \\
     6  &  $\SI{1.048(33)}{}$  \\
     7  &  $\SI{0.688(22)}{}$  \\
     8  &  $\SI{0.438(17)}{}$  \\
    10  &  $\SI{0.298(14)}{}$  \\
    12  &  $\SI{0.133(9)}{}$  \\
    15  &  $\SI{0.048(5)}{}$  \\
    18  &  $\SI{0.020(3)}{}$  \\
    20  &  $\SI{0.014(3)}{}$  \\
      \bottomrule
  \end{tabular}
\end{table}

Die Fehler werden hier nach der Gauß'schen Fehlerfortpflanzung bestimmt:
\begin{equation*}
  \sigma_{\frac{\symup{d}\sigma}{\symup{d}\Omega}} = \sqrt{\left(\frac{\sigma_N}{t \cdot I_0 \cdot N_\symup{Au} \cdot \Delta x \cdot \Delta \Omega}\right)^2
  + \left(\frac{N \cdot \sigma_{I_0}}{t \cdot I_0^2 \cdot N_\symup{Au} \cdot \Delta x \cdot \Delta \Omega}\right)^2}
\end{equation*}

Da es sich bei dem Versuch um ein Zählexperiment handelt, ergeben sich die Fehler der Zählraten jeweils aus der Wurzel der gezählten Ereignisse.

Die berechneten Werte und die Theoriekurve, welche sich aus Gleichung \ref{eqn:rutherford} ergibt, werden in Abbildung \ref{fig:wirkungsquerschnitt} dargestellt.

\begin{figure}[H]
  \centering
  \includegraphics{build/winkel.pdf}
  \caption{Berechneter und gemessener differentieller Wirkungsquerschnitt}
  \label{fig:wirkungsquerschnitt}
\end{figure}


\subsection{Untersuchung der Mehrfachstreuung}
Die Messung mit einer Goldfolie der Dicke $\Delta x = \SI{4}{\micro\meter}$ über einen
Zeitraum von $t = \SI{300}{\second}$ bei einem Winkel von $\Theta = 2 \, °$ ergibt
eine Zählrate von
\begin{equation*}
  N_\symup{4\mu m} = \SI{1033(32)}{1\per\second}
\end{equation*}
Daraus lässt sich entsprechend Gleichung \ref{eqn:wq} wieder der differentielle Wirkungsquerschnitt
berechnen. In Tabelle \ref{tab:wq} sind nun die Wirkungsquerschnitte der $\SI{2}{\micro\meter}$-Folie
und der $\SI{4}{\micro\meter}$-Folie bei dem entsprechenden Winkel von $\Theta = 2 \, °$ aufgeführt.
\begin{table}[H]
  \centering
  \caption{Differentielle Wirkungsquerschnitte der verschiedenen Goldfolien bei einem Winkel von $\Theta = 2 \, °$}
  \label{tab:wq}
  \begin{tabular}{c c}
    \toprule
    $\frac{\symup{d}\sigma}{\symup{d}\Omega}_\symup{2\mu m} /  10^{-23}\, \mathrm{m}$ & $\frac{\symup{d}\sigma}{\symup{d}\Omega}_\symup{4\mu m}/ 10^{-23}\, \mathrm{m}$ \\
    \midrule
    $\SI{30.08(81)}{}$ & $\SI{5.93(20)}{}$  \\
    \bottomrule
  \end{tabular}
\end{table}


\subsection{Untersuchung der Z-Abhängigkeit}
Die untersuchten Elemente werden mit ihrer gemessenen Zählrate $I$, der Ordnungszahl Z und der entsprechenden
Foliendicke $\Delta x$ in Tabelle \ref{tab:stoffe} aufgeführt. Des Weiteren sind die Dichte $\rho$ sowie die
molare Masse $M$ der entsprechenden Stoffe und der daraus berechenbare Term $\frac{I}{N\cdot \Delta x}$
angegeben. $N$ bezeichnet dabei die in Abschnitt \ref{sec:wq} bereits erwähnte Teilchendichte des entsprechenden Elements.
Sie berechnet sich auch hier nach Gleichung \ref{eqn:teilchendichte} für die jeweiligen Stoffe.
Der Winkel beträgt dabei $\Theta = 2 \, °$.
\begin{table}[H]
  \centering
  \caption{Spezifische Angaben und Zählraten der untersuchten Folien bei der Untersuchung der Z-Abhängigkeit \cite{sample}}
  \label{tab:stoffe}
  \begin{tabular}{c c c c c c c}
    \toprule
     & Z & $\Delta x/ \mathrm{\mu m}$ & I / $\frac{1}{\mathrm{s}}$ & $\rho/ \mathrm{\frac{g}{cm^3}}$ & M/$\mathrm{\frac{g}{mol}}$ & $\frac{I}{N\cdot \Delta x}$/$ 10^{-23}\ \mathrm{\frac{m^2}{s}}$ \\
    \midrule
    Gold & 79 & 4 & $\SI{3,44(11)}{}$ & 19,32 & 196.97 & $\SI{1.46(5)}{}$ \\
    Aluminium & 13 & 3 & $\SI{14,16(22)}{}$ & 2,70 & 26.98 & $\SI{7.83(12)}{}$ \\
    Bismut & 83 & 1 & $\SI{12,14(20)}{}$ & 9,80 & 208.98 & $\SI{42.99(71)}{}$ \\
    \bottomrule
  \end{tabular}
\end{table}

In Abbildung \ref{fig:zmessung} sind dann die berechneten Werte für $\frac{I}{N\cdot \Delta x}$
gegen die Ordnungszahl Z der jeweiligen Elemente aufgetragen.

\begin{figure}[H]
  \centering
  \includegraphics{build/z.pdf}
  \caption{$\frac{I}{N\cdot \Delta x}$ in Abhängigkeit der jeweiligen Ordnungszahl der Elemente.}
  \label{fig:zmessung}
\end{figure}

Die Fehler werden hier ganz einfach nach
\begin{equation*}
  \sigma_{\frac{I}{N\cdot \Delta x}} = \frac{\sigma_I}{N\cdot \Delta x}
\end{equation*}
berechnet.

\section{Diskussion}
\label{sec:Diskussion}

Die berechneten Werte für den differentiellen Wirkungsquerschnitt folgen außer in
einem Bereich kleiner Winkel (circa 0°-2°) in etwa der theoretisch erwarteten Kurve
der Rutherford-Formel. Jedoch liegen die einzelnen Werte ungefähr eine Größenordnung
über den Theoretischen. Fehlerquellen können dabei statistische sowie auch systematische
Unsicherheiten sein. Die statistischen Unsicherheiten sollten sich dabei aber vorrangig bei
den Werten großer Winkel auswirken. In diesem Bereich ist es nämlich nicht mehr möglich
ausreichend hohe Zählraten zu messen, sodass der statistische Fehler des poissonverteilten
Zählexperiments minimiert werden würde. Bei kleinen Winkeln hingegen sind die Zählraten hinreichend
groß. Es müssen sich folglich auch systematische Fehler einwirken. Diese könnten
zum einen aus einem nicht verschwindenden Kammerdruck sowie auch aus kleinen Druckschwankungen
resultieren. Des weiteren ist es möglich, dass die Geometrie des Aufbaus nicht vollkommen der
theoretisch angenommenen Geometrie entspricht. Einzelne Abstände könnten z.B. leicht abweichen.

Bei der Untersuchung der Mehrfachstreuung ist zu erkennen, dass sich der differentielle Wirkungsquerschnitt
bei einer dickeren Folie verringert. Dies ist dadurch zu erklären, dass sich durch die
Mehrfachstreuung die vielen Ablenkungen um kleinen Winkel zu großen Ablenkungen aufsummieren.
Entsprechend sollten die die differentiellen Wirkungsquerschnitte bei größeren Winkeln zunehmen.
Somit folgen die Messungen den theoretischen Erwartungen.

Die gemessenen Werte der Untersuchung der Z-Abhängigkeit zeigen, dass der Term $\frac{I}{N\cdot \Delta x}$
für Gold am kleinsten ist, und für Bismut am größten. Der Term wirkt sich, wie an
Gleichung \ref{eqn:wq} zu erkennen, direkt in den differentiellen Wirkungsquerschnitt ein.
Dieser sollte laut der Rutherford-Streuformel für größere Ordungszahl wachsen.
Folglich entspricht es also nicht den theoretischen Erwartungen, dass der berechnete
Term für Gold am geringsten ist. Dieser sollte nämlich für Aluminium am geringsten sein.
Für Bismut ist er wie erwartet am größten. Grund für die Abweichungen könnten aufgrund
der hier hinreichend großen Zählraten die oben bereits diskutierten systematischen
Fehler sein.

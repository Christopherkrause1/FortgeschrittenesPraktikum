\section{Theorie}
\label{sec:Theorie}

Durchdringen $\alpha$-Teilchen durch Materie, kommt es zu zwei unterschiedlichen Wechselwirkungen.
Zum einen können die $\alpha$-Teilchen Energie an die Hüllenelektronen abgeben und zum anderen
an Atomkernen streuen. Der Energieverlust pro Weglänge $x$ der $\alpha$-Teilchen durch die Hüllenelektronen wird dabei für
kleine Geschwindigkeiten durch die
Bethe-Bloch-Gleichung beschrieben.
\begin{align}
  -\frac{\mathrm{d}E}{\mathrm{d}x} =-\frac{\Delta E}{\Delta x}= - \frac{4\pi e^2 z^2 N Z}{m_{\mathrm{0}}v^2(4 \pi \epsilon_{\mathrm{0}})^2} \ln{\frac{2 m_{\mathrm{0}} v^2}{I}}
\end{align}
Mit der Elektronendichte $N$, der Kernladungszahl $Z$, der Geschwindigkeit $v$, der mittleren Ionisationsenergie $I$ des Materials
und der Ruhenergie $m_{\mathrm{0}}$ der Elektronen. Diese Formel ist nur unter den Annahmen eines einfallenden
schweren Teilchens und in ruhe befindenden Elektronen gültig. Zusätzlich wird angenommen, dass das einfallende Teilchen nicht
abgelenkt wird.

Die Streuung der Teilchen an einem Atomkern wird durch die Rutherfordsche Streuformel beschrieben. Hierbei gilt:
\begin{align}
  \frac{\mathrm{d}\sigma}{\mathrm{d}\Omega}(\Theta) = \frac{1}{(4 \pi \epsilon_{\mathrm{0}})^2} \left(\frac{z Z e^2}{4 E_{\mathrm{\alpha}}}\right)^2 \frac{1}{\sin^4{\frac{\Theta}{2}}}
  \label{eqn:rutherford}
\end{align}
Hierbei ist $\frac{\mathrm{d}\sigma}{\mathrm{d}\Omega}(\Theta)$ der differentielle Wirkungsquerschnitt in Abhängigkeit von dem Winkel $\Theta$
zwischen einfallendem und gestreuten $\alpha$-Teilchen und
$E_{\mathrm{\alpha}}$ die mittlere kinetische Energie der $\alpha$-Teilchen.
Hierbei wird angenommen, dass das Targetteilchen keine Ausdehnung hat und wesentlich schwerer
als das einfallende, spinlose Teilchen ist. Der Stoß ist elastisch.

Der Wirkungsquerschnitt beschreibt die Wahrscheinlichkeit eines einfallenden Teilchen an
einem Target zu streuen. Kommt es bei der Streuung zur Richtungsänderung, so wird
die Intensitätsverteilung des gestreuten Teilchens durch den differetiellen Wirkungsquerschnitt
beschrieben.

Americium ist ein $\alpha$- und ein $\beta$-Strahler. Der $\alpha$-Zerfall von ${}^{241}\mathrm{Am}$ sieht wie folgt aus:
\begin{align*}
  {}^{241}\mathrm{Am} \rightarrow {}^{237}\mathrm{Np}
\end{align*}

Die gemessene Zählrate ist poissonverteilt, wodurch der Fehler die Wurzel der Zählrate ist. Um den
statistischen Fehler kleiner als $3\%$ zu halten muss die Zählrate mindestens 1112 betragen.

\section{Theorie}
\label{sec:Theorie}
%\cite{sample}
Durchdringen $\alpha$-Teilchen durch Materie, kommt es zu zwei unterschiedlichen Wechselwirkungen.
Zum einen können die $\alpha$-Teilchen Energie an die Hüllenelektronen abgeben und zum anderen
an Atomkernen streuen. Der Energieverlust pro Weglänge $x$ der $\alpha$-Teilchen durch die Hüllenelektronen wird dabei für
kleine Geschwindigkeiten durch die
Bethe-Bloch-Gleichung beschrieben.
\begin{align}
  -\frac{\mathrm{d}E}{\mathrm{d}x} =-\frac{\Delta E}{\Delta x}= - \frac{4\pi e^2 z^2 N Z}{m_{\mathrm{0}}v^2(4 \pi \epsilon_{\mathrm{0}})^2} \ln{\frac{2 m_{\mathrm{0}} v^2}{I}}
\end{align}
Mit der Elektronendichte $N$, der Kernladungszahl $Z$, der Geschwindigkeit $v$, der mittleren Ionisationsenergie $I$ des Materials
und der Ruhenergie $m_{\mathrm{0}}$ der Elektronen.

Die Streuung der Teilchen an einem Atomkern wird durch die Rutherfordsche Streuformel beschrieben. Hierbei gilt:
\begin{align}
  \frac{\mathrm{d}\sigma}{\mathrm{d}\Omega}(\Theta) = \frac{1}{(4 \pi \epsilon_{\mathrm{0}})^2} \left(\frac{z Z e^2}{4 E_{\mathrm{\alpha}}}\right)^2 \frac{1}{\sin^4{\frac{\Theta}{2}}}
  \label{eqn:rutherford}
\end{align}
Hierbei ist $\frac{\mathrm{d}\sigma}{\mathrm{d}\Omega}(\Theta)$ der differentielle Wirkungsquerschnitt in Abhängigkeit von dem Winkel $\Theta$
zwischen einfallendem und gestreuten $\alpha$-Teilchen und
$E_{\mathrm{\alpha}}$ die mittlere kinetische Energie der $\alpha$-Teilchen.

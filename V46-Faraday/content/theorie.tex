\section{Theorie}
\label{sec:Theorie}

Der Faraday-Effekt beschreibt die Drehung der Polarisationsebene eines linear
polarisierten Lichtstrahl bei der Transmission durch ein Medium, wenn ein
externes Magnetfeld parallel zur Ausbreitungsrichtung des Strahls vorhanden
ist. Eine linear polarisierte Welle kann als die überlagerung von zwei
zirkular polarisierten Wellen verstanden werden, welche unterschiedliche
Phasengeschwindigkeiten haben.

In einem optisch aktiven Medium findet der Faraday-Effekt, auch zirkuläre Doppelbrechung genannt, über induzierte
Dipole durch die EM-Welle statt. Die dadurch hervorgerufene Polarisation
\begin{align}
  \vec{P} = \epsilon_0 \chi \vec{E}
\end{align}
ist proportional zum elektrischen Feld $\vec{E}$. Hierbei ist $\epsilon_0$ die
Influenzkonstante und $\chi $ die dielektrische Suszeptibilität. In anisotropen
Kristallen ist $\chi$ ein Tensor. Materie ist genau dann Doppelbrechend, wenn der in dem
Tensor nicht-diagonale und komoplex konjugierte Koeffizenten auftreten. Der Tensor
hat im einfachsten Fall dann die Gestalt:
\begin{align}
  \chi =
  \left( \begin{matrix}
         \chi_{\mathrm{xx}} & -i \chi_{\mathrm{xy}} & 0 \\
         -i \chi_{\mathrm{yx}} & \chi_{\mathrm{xx}} & 0 \\
         0 & 0 & \chi_{\mathrm{zz}}  \\
  \end{matrix} \right)
\end{align}

Für optisch inaktive Materie hat der Tensor nur diagonal Elemente. Wird ein
externes Magnetfeld parallel zur Ausbreitungsrichtung angelegt, verändert sich der Tensor und erhält
nicht-diagonal Elemente, wodurch die optisch inaktive Materie doppelbrechend wird.

Für den Rotationswinkel $\Theta$ um der sich die Polarisationsebene dreht gilt:
\begin{align}
  \Theta \approx \frac{L \omega}{2 c^2}v_{\mathrm{Ph}} \chi_{\mathrm{xy}} = \frac{L \omega}{2 c}\chi_{\mathrm{xy}}
\end{align}

Für Messfrequenzen von $\omega = 10°{14}-10^{15}$\, Hz gilt:
\begin{align}
  \Theta \approx \frac{e^3_0}{2 \epsilon_0 c} \frac{1}{m^2}\frac{\omega^2}{\omega^2_0 - \omega^2}^2 \frac{NBL}{n}
\end{align}


Über die effektive Masse können Information der Bandstruktur erhalten werden.
Um die effektive Masse zu bestimmen wird die Elektronenergie $\epsilon (k)$,
welche die Form der Bandkante beschreibt, in eine Taylor-Reihe entwickelt.

\begin{align}
  \epsilon(k) = \epsilon (0) + \frac{1}{2} \sum_{i=1}^3 \left(\frac{\partial \epsilon^2}{\partial k^2_{\mathrm{i}}}\right)_{k=0} k_{\mathrm{i}}^2 + ...
\end{align}

Wird dies mit Verglichen mit
\begin{align}
  \epsilon = \frac{\hbar^2 k^2}{2m}
\end{align}
so folgt für die effektive Masse $m^*$:
\begin{align}
  m^*_{\mathrm{i}} := \frac{\hbar}{\left(\frac{\partial \epsilon^2}{\partial k^2_{\mathrm{i}}}\right)_{k=0}}
\end{align}

Mit der effektiven Masse können Elektronen in einem Band mit kugelförmigen
Energieflächen im k-Raum wie freie Teilchen behandeln.

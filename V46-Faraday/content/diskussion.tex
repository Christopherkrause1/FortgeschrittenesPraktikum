\section{Diskussion}
\label{sec:Diskussion}
Die Messwerte der reinen GaAs-Probe stimmen mit den Erwartungen überein. Der
Drehwinkel sollte mit steigender Wellenlänge fallen, was im aufgezeigten Verlauf
deutlich zu erkennen ist. Trotzdem ergeben sich bei den Differenzen der reinen
und der dotierten Proben keine eindeutig linearen Verläufe, wie es nach der
Theorie zu erwarten wäre. Auch weichen die daraus berechneten effektiven Massen
$m^*_1$ und $m^*_2$ mit ca. $11,25\%$ merklich voneinander ab, obwohl sie den
gleichen Wert beschreiben sollten. Grund für die beschriebenen Abweichungen werden
im wesentlichen systematische Fehler sein. Es ist nicht möglich, die Spannung am
Differenzverstärker eindeutig auf das Minimum abzuregeln. Hinzu kommt, dass das Signal
trotz des eingebauten Selektivverstärkers immernoch Rauschspannung enthält. Die
Messung des Magnetfelds findet nicht exakt an der gleichen Stelle statt, an welcher
sich die Probe während der Messung befindet. Auch die Justierung
der gesamten Apparatur selbst ist nicht mit höchster Genauigkeit realisierbar.

\section{Auswertung}
\label{sec:Auswertung}

Die Magnetfeldstärke an der Stelle der Probe wird gemessen als
\begin{equation*}
  B = \SI{421}{\milli\tesla}.
\end{equation*}

Die gemessenen Winkel $\Theta_l$ und $\Theta_r$ sowie der daraus nach Gleichung
\ref{eqn:drehwinkel} berechnete gesamte Drehwinkel $\Theta$ und der auf die Länge
normierte Drehwinkel $\Theta_\symup{frei}$ werden in Tabelle ... in Abhängigkeit
der Wellenlänge $\lambda$ für die reine Probe aufgeführt.

Die berechneten Werte für den auf die Länge normierten Drehwinkel werden
in Abbildung \ref{fig:rein} dargestellt.
\begin{figure}[H]
  \centering
  \includegraphics{rein.pdf}
  \caption{Der auf die Länge normierte Drehwinkel in Abhängigkeit der Wellenlänge für die reine Probe.}
  \label{fig:rein}
\end{figure}

Die gemessenen Winkel $\Theta_l$ und $\Theta_r$ sowie der daraus nach Gleichung
\ref{eqn:drehwinkel} berechnete gesamte Drehwinkel $\Theta$ und der auf die Länge
normierte Drehwinkel $\Theta_\symup{frei}$ werden in Tabelle ... in Abhängigkeit
der Wellenlänge $\lambda$ für die dotierte Probe ($N = \SI{1.2e18}{1\per\centi\meter^3}$) aufgeführt.

Die berechneten Werte für den auf die Länge normierten Drehwinkel werden
in Abbildung \ref{fig:rein} dargestellt.
\begin{figure}[H]
  \centering
  \includegraphics{dotiert136.pdf}
  \caption{Der auf die Länge normierte Drehwinkel in Abhängigkeit der Wellenlänge für die reine Probe.}
  \label{fig:rein}
\end{figure}

Die gemessenen Winkel $\Theta_l$ und $\Theta_r$ sowie der daraus nach Gleichung
\ref{eqn:drehwinkel} berechnete gesamte Drehwinkel $\Theta$ und der auf die Länge
normierte Drehwinkel $\Theta_\symup{frei}$ werden in Tabelle ... in Abhängigkeit
der Wellenlänge $\lambda$ für die dotierte Probe ($N = \SI{2.8e18}{1\per\centi\meter^3}$) aufgeführt.

Die berechneten Werte für den auf die Länge normierten Drehwinkel werden
in Abbildung \ref{fig:rein} dargestellt.
\begin{figure}[H]
  \centering
  \includegraphics{dotiert1296.pdf}
  \caption{Der auf die Länge normierte Drehwinkel in Abhängigkeit der Wellenlänge für die reine Probe.}
  \label{fig:rein}
\end{figure}

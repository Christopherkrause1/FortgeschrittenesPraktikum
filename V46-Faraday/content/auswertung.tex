\section{Auswertung}
\label{sec:Auswertung}
\subsection{Bestimmung der Faraday-Rotation pro Einheitslänge in Abhängigkeit von der Wellenlänge}
Die Magnetfeldstärke an der Stelle der Probe wird gemessen als
\begin{equation*}
  B = \SI{421}{\milli\tesla}.
\end{equation*}

Die gemessenen Winkel $\Theta_l$ und $\Theta_r$ sowie der daraus nach Gleichung
\ref{eqn:drehwinkel} berechnete gesamte Drehwinkel $\Theta$ und der auf die Länge
normierte Drehwinkel $\Theta_\symup{frei}$ werden in Tabelle \ref{tab:rein} in Abhängigkeit
der Wellenlänge $\lambda$ für die reine Probe aufgeführt.

\begin{table}[H]
  \centering
  \caption{Wellenlängen der Interferenzfilter und gemessene Winkel für die reine GaAs Probe}
  \label{tab:rein}
  \begin{tabular}{c c c c c}
    \toprule
    $\lambda/\mathrm{\mu m}$ & $\Theta_{\mathrm{L}}/$rad & $\Theta_{\mathrm{R}}/$rad  & $\Theta /$rad & $\Theta_{\mathrm{frei}}/$rad\\
    \midrule
    1,06  & 0,2836 & 0,6472 & 0,1818 & 35,6480 \\
    1,29  & 0,3054 & 0,6097 & 0,1521 & 29,8303 \\
    1,45  & 0,3601 & 0,5745 & 0,1071 & 21,0180 \\
    1,72  & 0,4028 & 0,5262 & 0,0616 & 12,0918 \\
    1,96  & 0,4209 & 0,5273 & 0,0532 & 10,4377 \\
    2,16  & 0,4354 & 0,5235 & 0,0440 & 8,64109 \\
    2,34  & 0,4372 & 0,5009 & 0,0318 & 6,24554 \\
    2,51  & 0,4782 & 0,4898 & 0,0058 & 1,14073 \\
    2,65  & 0,4694 & 0,4878 & 0,0091 & 1,79666 \\
    \bottomrule
  \end{tabular}
\end{table}



Die berechneten Werte für den auf die Länge normierten Drehwinkel werden
in Abbildung \ref{fig:rein} dargestellt.
\begin{figure}[H]
  \centering
  \includegraphics{rein.pdf}
  \caption{Auf die Länge normierter Drehwinkel in Abhängigkeit der Wellenlänge für die reine Probe.}
  \label{fig:rein}
\end{figure}

Die gemessenen Winkel $\Theta_l$ und $\Theta_r$ sowie der daraus nach Gleichung
\ref{eqn:drehwinkel} berechnete gesamte Drehwinkel $\Theta$ und der auf die Länge
normierte Drehwinkel $\Theta_\symup{frei}$ werden in Tabelle \ref{tab:erste} in Abhängigkeit
der Wellenlänge $\lambda$ für die dotierte Probe ($N = \SI{1.2e18}{1\per\centi\meter^3}$) aufgeführt.

\begin{table}[H]
  \centering
  \caption{Wellenlängen der Interferenzfilter und gemessene Winkel für die erste n-dotierte GaAs Probe}
  \label{tab:erste}
  \begin{tabular}{c c c c c}
    \toprule
    $\lambda/\mathrm{\mu m}$ & $\Theta_{\mathrm{L}}/$rad & $\Theta_{\mathrm{R}}/$rad  & $\Theta /$rad & $\Theta_{\mathrm{frei}}/$rad\\
    \midrule
    1,06  & 0,3665 & 0,5430 & 0,0882 & 64,91 \\
    1,29  & 0,3737 & 0,5329 & 0,0795 & 58,49 \\
    1,45  & 0,4043 & 0,4991 & 0,0474 & 34,86 \\
    1,72  & 0,3871 & 0,4770 & 0,0449 & 33,04 \\
    1,96  & 0,3787 & 0,5177 & 0,0695 & 51,11 \\
    2,16  & 0,3447 & 0,4788 & 0,0670 & 49,30 \\
    2,34  & 0,4084 & 0,5061 & 0,0488 & 35,93 \\
    2,51  & 0,3784 & 0,4898 & 0,0557 & 40,95 \\
    2,65  & 0,4523 & 0,5235 & 0,0356 & 26,20 \\
    \bottomrule
  \end{tabular}
\end{table}



Die berechneten Werte für den auf die Länge normierten Drehwinkel werden
in Abbildung \ref{fig:rein} dargestellt.
\begin{figure}[H]
  \centering
  \includegraphics{dotiert136.pdf}
  \caption{Auf die Länge normierter Drehwinkel in Abhängigkeit der Wellenlänge für die dotierte Probe ($N = \SI{1.2e18}{1\per\centi\meter^3}$).}
  \label{fig:rein}
\end{figure}

Die gemessenen Winkel $\Theta_l$ und $\Theta_r$ sowie der daraus nach Gleichung
\ref{eqn:drehwinkel} berechnete gesamte Drehwinkel $\Theta$ und der auf die Länge
normierte Drehwinkel $\Theta_\symup{frei}$ werden in Tabelle \ref{tab:zweite} in Abhängigkeit
der Wellenlänge $\lambda$ für die dotierte Probe ($N = \SI{2.8e18}{1\per\centi\meter^3}$) aufgeführt.

\begin{table}[H]
  \centering
  \caption{Wellenlängen der Interferenzfilter und gemessene Winkel für die zweite n-dotierte GaAs Probe}
  \label{tab:zweite}
  \begin{tabular}{c c c c c}
    \toprule
    $\lambda/\mathrm{\mu m}$ & $\Theta_{\mathrm{L}}/$rad & $\Theta_{\mathrm{R}}/$rad  & $\Theta /$rad & $\Theta_{\mathrm{frei}}/$rad\\
    \midrule
    1,06  & 0,4380 & 0,4593 & 0,0106 & 8,192 \\
    1,29  & 0,3746 & 0,5285 & 0,0769 & 59,36 \\
    1,45  & 0,3708 & 0,5119 & 0,0705 & 54,42 \\
    1,72  & 0,3676 & 0,5090 & 0,0706 & 54,54 \\
    1,96  & 0,3505 & 0,5430 & 0,0962 & 74,29 \\
    2,16  & 0,3415 & 0,5657 & 0,1121 & 86,52 \\
    2,34  & 0,3860 & 0,5209 & 0,0674 & 52,07 \\
    2,51  & 0,3490 & 0,4843 & 0,0676 & 52,18 \\
    2,65  & 0,3255 & 0,5637 & 0,1191 & 91,91 \\
    \bottomrule
  \end{tabular}
\end{table}


Die berechneten Werte für den auf die Länge normierten Drehwinkel werden
in Abbildung \ref{fig:rein} dargestellt.
\begin{figure}[H]
  \centering
  \includegraphics{dotiert1296.pdf}
  \caption{Auf die Länge normierter Drehwinkel in Abhängigkeit der Wellenlänge für die dotierte Probe ($N = \SI{2.8e18}{1\per\centi\meter^3}$).}
  \label{fig:rein}
\end{figure}


\subsection{Bestimmung der effektiven Masse}
Für den Brechungsindex wird ein Mittelwert aus den verschiedenen Brechungsindices,
welche in dem Wellenlängenbereich des hier zu betrachtenden Lichtes liegen, berechnet. In Abbildung \ref{tab:Brechung}
sind die einzelnen Brechungsindices dargestellt.

\begin{table}[H]
  \centering
  \caption{Brechungsindices von Galliumarsenid}
  \label{tab:Brechung}
  \begin{tabular}{c c}
    \toprule
    $\lambda/\mathrm{\mu m}$ & n\\
    \midrule
    1,033  & 3,492 \\
    1,127  & 3,455 \\
    1,240  & 3,423 \\
    1,378  & 3,397 \\
    1,550  & 3,374 \\
    1,771  & 3,352 \\
    2,066  & 3,338 \\
    \bottomrule
  \end{tabular}
\end{table}

Der Mittelwert von $n$ beträgt:
\begin{align*}
  \bar{n} = \SI{3.40(5)}{}
\end{align*}
Der Fehler wird dabei wie folgt berechnet.
\begin{align*}
  \sigma_n = \sqrt{\frac{1}{N\cdot(N-1)} \sum_i (n_i - \bar{n})^2}
\end{align*}

In den Abbildungen \ref{fig:differenz136} und \ref{fig:differenz1296} ist eine lineare
Anpassungsfunktion $y=a \cdot x$ für die Werte von $\Theta_{\mathrm{frei}}$ dargestellt.

\begin{figure}[H]
  \centering
  \includegraphics{differenz136.pdf}
  \caption{Differenz der auf die Länge normierten Drehwinkel der dotierten Probe ($N = \SI{1.2e18}{1\per\centi\meter^3}$) und der reinen Probe in Abhängigkeit der Wellenlänge.}
  \label{fig:differenz136}
\end{figure}

Für den Parameter $a$ ergibt sich für die erste Probe:
\begin{align*}
  a = \SI{6.4(11)}{\radian\per\meter^3}
\end{align*}
Daraus wird mit Gleichung \ref{eqn:frei} die effektive Masse bestimmt.
\begin{align*}
  m^*_1 = \sqrt{\frac{NB e^3_0}{8 \pi^2 \epsilon_0 n c^3 a}} = \SI{7.1(6)e-32}{\kilo\gram}
\end{align*}


\begin{figure}[H]
  \centering
  \includegraphics{differenz1296.pdf}
  \caption{Differenz der auf die Länge normierten Drehwinkel der dotierten Probe ($N = \SI{2.8e18}{1\per\centi\meter^3}$) und der reinen Probe in Abhängigkeit der Wellenlänge.}
  \label{fig:differenz1296}
\end{figure}


Für den Parameter $a$ ergibt sich für die zweite Probe:
\begin{align*}
  a = \SI{11.7(16)}{\radian\per\meter^3}
\end{align*}
Daraus wird die effektive Masse bestimmt.
\begin{align*}
  m^*_2 = \sqrt{\frac{NB e^3_0}{8 \pi^2 \epsilon_0 n c^3 a}} = \SI{8.0(6)e-32}{\kilo\gram}
\end{align*}


Der Fehler der effektiven Masse wird dabei mit der Gaußschen Fehlerfortpflanzung berechnet.
\begin{align*}
  \sigma_{m^*} &= \sqrt{\frac{\gamma^{2} \sigma_{n}^{2}}{4 a n^{3}}  + \frac{\gamma^{2} \sigma_{a}^{2}}{4 a^{3} n}} \\
  \text{mit}\:\: \gamma &= \sqrt{\frac{NB e^3_0}{8 \pi^2 \epsilon_0 c^3}}
\end{align*}

\section{Auswertung}
\label{sec:Auswertung}

Die Magnetfeldstärke an der Stelle der Probe wird gemessen als
\begin{equation*}
  B = \SI{421}{\milli\tesla}.
\end{equation*}

Die gemessenen Winkel $\Theta_l$ und $\Theta_r$ sowie der daraus nach Gleichung
\ref{eqn:drehwinkel} berechnete gesamte Drehwinkel $\Theta$ und der auf die Länge
normierte Drehwinkel $\Theta_\symup{frei}$ werden in Tabelle ... in Abhängigkeit
der Wellenlänge $\lambda$ für die reine Probe aufgeführt.

\begin{table}[H]
  \centering
  \caption{Wellenlängen der Intereferenzfilter und gemessene Winkel für die reine GaAs Probe}
  \label{tab:Parameter}
  \begin{tabular}{c c c c c}
    \toprule
    $\lambda/\mathrm{\mu m}$ & $\Theta_{\mathrm{L}}/$rad & $\Theta_{\mathrm{R}}/$rad  & $\Theta /$rad & $\Theta_{\mathrm{frei}}/$rad\\
    \midrule
    1,06  & 0,2836 & 0,6472 & 0,1818 & 35,6480 \\
    1,29  & 0,3054 & 0,6097 & 0,1521 & 29,8303 \\
    1,45  & 0,3601 & 0,5745 & 0,1071 & 21,0180 \\
    1,72  & 0,4028 & 0,5262 & 0,0616 & 12,0918 \\
    1,96  & 0,4209 & 0,5273 & 0,0532 & 10,4377 \\
    2,16  & 0,4354 & 0,5235 & 0,0440 & 8,64109 \\
    2,34  & 0,4372 & 0,5009 & 0,0318 & 6,24554 \\
    2,51  & 0,4782 & 0,4898 & 0,0058 & 1,14073 \\
    2,65  & 0,4694 & 0,4878 & 0,0091 & 1,79666 \\
    \bottomrule
  \end{tabular}
\end{table}



Die berechneten Werte für den auf die Länge normierten Drehwinkel werden
in Abbildung \ref{fig:rein} dargestellt.
\begin{figure}[H]
  \centering
  \includegraphics{rein.pdf}
  \caption{Der auf die Länge normierte Drehwinkel in Abhängigkeit der Wellenlänge für die reine Probe.}
  \label{fig:rein}
\end{figure}

Die gemessenen Winkel $\Theta_l$ und $\Theta_r$ sowie der daraus nach Gleichung
\ref{eqn:drehwinkel} berechnete gesamte Drehwinkel $\Theta$ und der auf die Länge
normierte Drehwinkel $\Theta_\symup{frei}$ werden in Tabelle ... in Abhängigkeit
der Wellenlänge $\lambda$ für die dotierte Probe ($N = \SI{1.2e18}{1\per\centi\meter^3}$) aufgeführt.

\begin{table}[H]
  \centering
  \caption{Wellenlängen der Intereferenzfilter und gemessene Winkel für die erste n-dotierte GaAs Probe}
  \label{tab:Parameter}
  \begin{tabular}{c c c c c}
    \toprule
    $\lambda/\mathrm{\mu m}$ & $\Theta_{\mathrm{L}}/$rad & $\Theta_{\mathrm{R}}/$rad  & $\Theta /$rad & $\Theta_{\mathrm{frei}}/$rad\\
    \midrule
    1.06  & 0.3665 & 0.5430 & 0.0882 & 64.91 \\
    1.29  & 0.3737 & 0.5329 & 0.0795 & 58.49 \\
    1.45  & 0.4043 & 0.4991 & 0.0474 & 34.86 \\
    1.72  & 0.3871 & 0.4770 & 0.0449 & 33.04 \\
    1.96  & 0.3787 & 0.5177 & 0.0695 & 51.11 \\
    2.16  & 0.3447 & 0.4788 & 0.0670 & 49.30 \\
    2.34  & 0.4084 & 0.5061 & 0.0488 & 35.93 \\
    2.51  & 0.3784 & 0.4898 & 0.0557 & 40.95 \\
    2.65  & 0.4523 & 0.5235 & 0.0356 & 26.20 \\
    \bottomrule
  \end{tabular}
\end{table}



Die berechneten Werte für den auf die Länge normierten Drehwinkel werden
in Abbildung \ref{fig:rein} dargestellt.
\begin{figure}[H]
  \centering
  \includegraphics{dotiert136.pdf}
  \caption{Der auf die Länge normierte Drehwinkel in Abhängigkeit der Wellenlänge für die dotierte Probe ($N = \SI{1.2e18}{1\per\centi\meter^3}$).}
  \label{fig:rein}
\end{figure}

Die gemessenen Winkel $\Theta_l$ und $\Theta_r$ sowie der daraus nach Gleichung
\ref{eqn:drehwinkel} berechnete gesamte Drehwinkel $\Theta$ und der auf die Länge
normierte Drehwinkel $\Theta_\symup{frei}$ werden in Tabelle ... in Abhängigkeit
der Wellenlänge $\lambda$ für die dotierte Probe ($N = \SI{2.8e18}{1\per\centi\meter^3}$) aufgeführt.

\begin{table}[H]
  \centering
  \caption{Wellenlängen der Intereferenzfilter und gemessene Winkel für die zweite n-dotierte GaAs Probe}
  \label{tab:Parameter}
  \begin{tabular}{c c c c c}
    \toprule
    $\lambda/\mathrm{\mu m}$ & $\Theta_{\mathrm{L}}/$rad & $\Theta_{\mathrm{R}}/$rad  & $\Theta /$rad & $\Theta_{\mathrm{frei}}/$rad\\
    \midrule
    1.06  & 0.4380 & 0.4593 & 0.0106 & 8.192 \\
    1.29  & 0.3746 & 0.5285 & 0.0769 & 59.36 \\
    1.45  & 0.3708 & 0.5119 & 0.0705 & 54.42 \\
    1.72  & 0.3676 & 0.5090 & 0.0706 & 54.54 \\
    1.96  & 0.3505 & 0.5430 & 0.0962 & 74.29 \\
    2.16  & 0.3415 & 0.5657 & 0.1121 & 86.52 \\
    2.34  & 0.3860 & 0.5209 & 0.0674 & 52.07 \\
    2.51  & 0.3490 & 0.4843 & 0.0676 & 52.18 \\
    2.65  & 0.3255 & 0.5637 & 0.1191 & 91.91 \\
    \bottomrule
  \end{tabular}
\end{table}


Die berechneten Werte für den auf die Länge normierten Drehwinkel werden
in Abbildung \ref{fig:rein} dargestellt.
\begin{figure}[H]
  \centering
  \includegraphics{dotiert1296.pdf}
  \caption{Der auf die Länge normierte Drehwinkel in Abhängigkeit der Wellenlänge für die dotierte Probe ($N = \SI{2.8e18}{1\per\centi\meter^3}$).}
  \label{fig:rein}
\end{figure}

\begin{figure}[H]
  \centering
  \includegraphics{differenz136.pdf}
  \caption{Differenz der auf die Länge normierten Drehwinkel der dotierten Probe ($N = \SI{1.2e18}{1\per\centi\meter^3}$) und der reinen Probe in Abhängigkeit der Wellenlänge.}
  \label{fig:differenz136}
\end{figure}

\begin{figure}[H]
  \centering
  \includegraphics{differenz1296.pdf}
  \caption{Differenz der auf die Länge normierten Drehwinkel der dotierten Probe ($N = \SI{2.8e18}{1\per\centi\meter^3}$) und der reinen Probe in Abhängigkeit der Wellenlänge.}
  \label{fig:differenz1296}
\end{figure}

\section{Auswertung}
\label{sec:Auswertung}


\subsection{Stabilitätsbedingung}
Der Abstand der Resonatoren voneinander und der zugehörige gemessene Strom wird in Tavelle 1 dargestellt.


\begin{table}[H]
\centering
\caption{Messdaten für dubiose Elemente.}
\sisetup{table-format=2.1}
\begin{tabular}{S S| S S}
  \toprule
    \multicolumn{2}{c}{Konkav/Planar} & \multicolumn{2}{c}{Konkav/Konkav} \\
    \cmidrule(lr){1-2}\cmidrule(lr){3-4}
    {$L/$cm} & {$I/\, \symup{\mu A}$} & {L/cm} & {$I/\, \symup{\mu A}$} \\
    \midrule
    60 &    120 &  60  &  266  \\
    62 &    112 &  63  &  342  \\
    63 &    113 &  66  &  382  \\
    64 &    102 &  69  &  398  \\
    66 &    130 &  72  &  353  \\
    68 &    128 &  75  &  367  \\
    69 &    111 &  78  &  332  \\
    70 &     70 &  81  &  385  \\
    72 &    116 &  84  &  391  \\
    74 &     69 &  87  &  222  \\
    75 &    105 &  90  &  120  \\
    76 &     65 &  93  &  215  \\
    78 &    112 &  96  &  211  \\
    81 &    111 &  99  &  112  \\
    84 &     77 &  102 &  146  \\
    87 &     45 &  105 &  157  \\
    90 &     39 &  108 &  176  \\
    93 &     22 &  111 &  164  \\
       &        &  114 &  203  \\
       &        &  117 &  240  \\
       &        &  120 &  241  \\
       &        &  123 &  205  \\
       &        &  126 &  209  \\
       &        &  129 &  225  \\
       &        &  132 &  217  \\
       &        &  135 &  255  \\
       &        &  138 &  276  \\
      \bottomrule
  \end{tabular}
\end{table}

In Abbildung 1 sind die Messwerte und die Anpassungsfunktion für 2 konkave Spiegel dargestellt.
Dabei hat die Anpassungsfunktion mit den Parameter $a$, $b$ und $c$ die Form:
\begin{align*}
  I_{\symup{1}} = a \cdot L^2 + b \cdot L + c
\end{align*}
\begin{figure}[H]
  \centering
  \includegraphics{stabilitaet.pdf}
  \caption{Messwerte und Anpassungsfunktion für zwei konkave Spiegel.}
  \label{fig:plot}
\end{figure}

Die Fitparameter betragen :
\begin{align*}
  a = \SI{ 8.0(25)e-2}{\milli\ampere\per\centi\meter\squared} \\
  b = \SI{-1.8(5)e1}{\milli\ampere\per\centi\meter} \\
  c = \SI{1.19(24)e3}{\milli\ampere}
\end{align*}


In Abbildung 2 sind die Messwerte und die Anpassungsfunktion für ein konkaven und einen
planaren Spiegel dargestellt.
Dabei hat die Anpassungsfunktion mit den Parametern $d$ und $e$ die Form:
\begin{align*}
  I_{\symup{1}} = d \cdot L + b
\end{align*}

\begin{figure}[H]
  \centering
  \includegraphics{stabilitaet_2.pdf}
  \caption{Messwerte und Anpassungsfunktion für ein konkaven und ein planaren Spiegel.}
  \label{fig:plot}
\end{figure}

Die Fitparameter betragen :
\begin{align*}
  &d = \SI{-2.6(5)}{\milli\ampere\per\centi\meter} \\
  &e = \SI{280(38)}{\milli\ampere}
\end{align*}


\subsection{Polarisation}
Der eingestellte Winkel am Polarisationsfilter und der zugehörige gemessene Strom
wird in Tabelle ... dargestellt.

In Abbildung \ref{fig:polarisation} sind die Messwerte und die entsprechende
Anpassungsfunktion aufgetragen. Bei der Anpassungsfunktion handelt es sich um eine
Funktion der Form:
\begin{equation*}
  I_\symup{P}(\phi_\symup{P}) = I_0 \cdot \cos^2(\phi_\symup{P} - \phi_0)
\end{equation*}

\begin{figure}[H]
  \centering
  \includegraphics{polarisation.pdf}
  \caption{Messwerte und Anpassungsfunktion der Polarisationsmessung.}
  \label{fig:polarisation}
\end{figure}

Die Fitparameter lauten:
\begin{align*}
  I_0 &= \SI{209.73(608)}{\micro\ampere} \\
  \phi_0 &= \SI{-74.98(3)}{}
\end{align*}

\section{Auswertung}
\label{sec:Auswertung}


\subsection{Stabilitätsbedingung}
Der Kurvenverlauf für die Stabilitätsbedingung in Abhängigkeit von dem Resonatorabstand wird in Abbildung 3
dargestellt.

\begin{figure}[H]
  \centering
  \includegraphics{parameter.pdf}
  \caption{Stabilitätsbedingung für konkave und planare Spiegel}
  \label{fig:plot}
\end{figure}

Der Abstand der Resonatoren voneinander und der zugehörige gemessene Strom wird in Tabelle 1 dargestellt.


\begin{table}[H]
\centering
\caption{Resonatorabstände und gemessene Stromstärken}
\sisetup{table-format=2.1}
\begin{tabular}{S S| S S}
  \toprule
    \multicolumn{2}{c}{Konkav/Planar} & \multicolumn{2}{c}{Konkav/Konkav} \\
    \cmidrule(lr){1-2}\cmidrule(lr){3-4}
    {$L/$cm} & {$I/\, \symup{\mu A}$} & {L/cm} & {$I/\, \symup{\mu A}$} \\
    \midrule
    60 &    120 &  60  &  266  \\
    62 &    112 &  63  &  342  \\
    63 &    113 &  66  &  382  \\
    64 &    102 &  69  &  398  \\
    66 &    130 &  72  &  353  \\
    68 &    128 &  75  &  367  \\
    69 &    111 &  78  &  332  \\
    70 &     70 &  81  &  385  \\
    72 &    116 &  84  &  391  \\
    74 &     69 &  87  &  222  \\
    75 &    105 &  90  &  120  \\
    76 &     65 &  93  &  215  \\
    78 &    112 &  96  &  211  \\
    81 &    111 &  99  &  112  \\
    84 &     77 &  102 &  146  \\
    87 &     45 &  105 &  157  \\
    90 &     39 &  108 &  176  \\
    93 &     22 &  111 &  164  \\
       &        &  114 &  203  \\
       &        &  117 &  240  \\
       &        &  120 &  241  \\
       &        &  123 &  205  \\
       &        &  126 &  209  \\
       &        &  129 &  225  \\
       &        &  132 &  217  \\
       &        &  135 &  255  \\
       &        &  138 &  276  \\
      \bottomrule
  \end{tabular}
\end{table}

In Abbildung 4 sind die Messwerte und die Anpassungsfunktion für 2 konkave Spiegel dargestellt.
Dabei hat die Anpassungsfunktion mit den Parameter $a$, $b$ und $c$ die Form:
\begin{align*}
  I_{\symup{1}} = a \cdot L^2 + b \cdot L + c
\end{align*}
\begin{figure}[H]
  \centering
  \includegraphics{stabilitaet.pdf}
  \caption{Messwerte und Anpassungsfunktion für zwei konkave Spiegel.}
  \label{fig:plot}
\end{figure}

Die Fitparameter betragen :
\begin{align*}
  a = \SI{ 8.0(25)e-2}{\milli\ampere\per\centi\meter\squared} \\
  b = \SI{-1.8(5)e1}{\milli\ampere\per\centi\meter} \\
  c = \SI{1.19(24)e3}{\milli\ampere}
\end{align*}


In Abbildung 5 sind die Messwerte und die Anpassungsfunktion für ein konkaven und einen
planaren Spiegel dargestellt.
Dabei hat die Anpassungsfunktion mit den Parametern $d$ und $e$ die Form:
\begin{align*}
  I_{\symup{1}} = d \cdot L + e
\end{align*}

\begin{figure}[H]
  \centering
  \includegraphics{stabilitaet_2.pdf}
  \caption{Messwerte und Anpassungsfunktion für ein konkaven und ein planaren Spiegel.}
  \label{fig:plot}
\end{figure}

Die Fitparameter betragen :
\begin{align*}
  &d = \SI{-2.6(5)}{\milli\ampere\per\centi\meter} \\
  &e = \SI{280(38)}{\milli\ampere}
\end{align*}

\subsection{TEM-Moden}
Die gemessenen Werte der Intensität der Grundmode und der ersten Mode bei den
entsprechenden Positionen der Diode werden in Tabelle 2 dargestellt.

\begin{table}[H]
\centering
\caption{Positionen der Photodiode und gemessene Stromstärken}
\sisetup{table-format=2.1}
\begin{tabular}{S S S S}
  \toprule
    \multicolumn{2}{c}{TEM$_{\symup{00}}$} & \multicolumn{2}{c}{TEM$_{\symup{01}}$} \\
    \cmidrule(lr){1-2}\cmidrule(lr){3-4}
    {Postion/mm} & {$I/\, \symup{\mu A}$} & {Position/mm} & {$I/\, \symup{\mu A}$} \\
    \midrule
    -15& 0.317 &    -15 & 0.135 \\
    -14& 0.363 &    -14 & 0.154 \\
    -13& 0.454 &    -13 & 0.176 \\
    -12& 0.526 &    -12 & 0.214 \\
    -11& 0.615 &    -11 & 0.237 \\
    -10& 0.719 &    -10 & 0.265 \\
    -9 & 0.790 &    -9  & 0.291 \\
    -8 & 0.896 &    -8  & 0.301 \\
    -7 & 0.962 &    -7  & 0.299 \\
    -6 & 1.023 &    -6  & 0.293 \\
    -5 & 1.066 &    -5  & 0.287 \\
    -4 & 1.079 &    -4  & 0.258 \\
    -3 & 1.094 &    -3  & 0.233 \\
    -2 & 1.102 &    -2  & 0.192 \\
    -1 & 1.070 &    -1  & 0.153 \\
    0  & 1.024 &    0   & 0.117 \\
    1  & 1.012 &    1   & 0.074 \\
    2  & 0.989 &    2   & 0.045 \\
    3  & 0.911 &    3   & 0.021 \\
    4  & 0.859 &    4   & 0.017 \\
    5  & 0.795 &    5   & 0.025 \\
    6  & 0.712 &    6   & 0.035 \\
    7  & 0.641 &    7   & 0.063 \\
    8  & 0.571 &    8   & 0.101 \\
    9  & 0.507 &    9   & 0.147 \\
    10 & 0.429 &    10  & 0.175 \\
    11 & 0.366 &    11  & 0.214 \\
       &       &    12  & 0.240 \\
       &       &    13  & 0.301 \\
       &       &    14  & 0.326 \\
       &       &    15  & 0.364 \\
       &       &    16  & 0.381 \\
       &       &    17  & 0.368 \\
       &       &    18  & 0.347 \\
       &       &    19  & 0.305 \\
       &       &    20  & 0.253 \\
       &       &    21  & 0.223 \\
       &       &    22  & 0.176 \\
       &       &    23  & 0.140 \\
       &       &    24  & 0.115 \\
       &       &    25  & 0.093 \\
    \bottomrule
  \end{tabular}
\end{table}

In Abbildung \ref{fig:grundmode} sind die Messwerte und die Anpassungsfunktion
für die Grundmode aufgetragen. Bei der Anpassungsfunktion handelt es sich dabei
um eine gaußsche Normalverteilung der Form:
\begin{equation*}
  I_\symup{G}(x) = I_0 \cdot exp\left(-2 \cdot \left(\frac{x - x_0}{\omega_0}\right)^2\right)
\end{equation*}

Bei der Variablen $x$ handelt es sich dabei um die Position der Diode auf der Schiene.

\begin{figure}[H]
  \centering
  \includegraphics{grundmode.pdf}
  \caption{Messwerte und Anpassungsfunktion Grundmode.}
  \label{fig:grundmode}
\end{figure}

Die Fitparameter lauten:
\begin{align*}
  I_0 &= \SI{1.10(1)}{\micro\ampere} \\
  x_0 &= \SI{-1.96(8)}{\milli\meter} \\
  \omega_0 &= \SI{17.04(19)}{\milli\meter}
\end{align*}

Entsprechend sind in Abbildung \ref{fig:erstemode} die Messwerte und Anpassungsfunktion
der ersten Mode aufgetragen. Bei der Anpassungsfunktion handelt es sich hierbei jeodch
nicht mehr um eine einfache Normalverteilung, sondern um eine Funktion der Form:
\begin{equation*}
  I_{1}(x) = I_1 \cdot \left(\frac{x - x_1}{\omega_1}\right)^2 \cdot exp\left(-2 \cdot \left(\frac{x - x_1}{\omega_1}\right)^2\right)
\end{equation*}

\begin{figure}[H]
  \centering
  \includegraphics{erstemode.pdf}
  \caption{Messwerte und Anpassungsfunktion der ersten Mode.}
  \label{fig:erstemode}
\end{figure}

Die Fitparameter lauten:
\begin{align*}
  I_1 &= \SI{1.78(3)}{\micro\ampere} \\
  x_1 &= \SI{4.35(12)}{\milli\meter} \\
  \omega_1 &= \SI{15.82(18)}{\milli\meter}
\end{align*}

\subsection{Polarisation}
Der eingestellte Winkel am Polarisationsfilter und der zugehörige gemessene Strom
wird in Tabelle 3 dargestellt.

\begin{table}[H]
\centering
\caption{Gemessene Winkel und Stromstärken}
\begin{tabular}{c c}
  \toprule
    {$\phi_{\symup{P}}/$°} & {$I/\, \symup{\mu A}$} \\
    \midrule
    0   &   186  \\
    20  &   204  \\
    40  &   204  \\
    60  &   121  \\
    80  &    46  \\
    100 &   6.8  \\
    120 &   4.4  \\
    140 &    44  \\
    160 &   126  \\
    180 &   178  \\
    200 &   222  \\
    220 &   200  \\
    240 &   138  \\
    260 &    59  \\
    280 &  17.6  \\
    300 &  0.31  \\
    320 &  13.4  \\
    340 &  66.9  \\
    \bottomrule
  \end{tabular}
\end{table}

In Abbildung \ref{fig:polarisation} sind die Messwerte und die entsprechende
Anpassungsfunktion aufgetragen. Bei der Anpassungsfunktion handelt es sich um eine
Funktion der Form:
\begin{equation*}
  I_\symup{P}(\phi_\symup{P}) = I_0 \cdot \cos^2(\phi_\symup{P} - \phi_0)
\end{equation*}

\begin{figure}[H]
  \centering
  \includegraphics{polarisation.pdf}
  \caption{Messwerte und Anpassungsfunktion der Polarisationsmessung.}
  \label{fig:polarisation}
\end{figure}

Die Fitparameter lauten:
\begin{align*}
  I_0 &= \SI{209.73(608)}{\micro\ampere} \\
  \phi_0 &= \SI{-74.98(3)}{}
\end{align*}

\subsection{Bestimmung der Wellenlänge}
Bei der Messung zur Bestimmung der Wellenlänge wurde der Abstand von Gitter zu
Detektorschirm gemessen als
\begin{equation*}
  L = \SI{143.8(1)}{\centi\meter}
\end{equation*}
Die Abstände zum linken und rechten Beugungsmaximum erster Ordnung betragen
\begin{align*}
  d_\symup{r} &= \SI{7.2(1)}{\centi\meter} \\
  d_\symup{l} &= \SI{7.1(1)}{\centi\meter}
\end{align*}
Daraus ergibt sich ein mittlerer Abstand zum ersten Beugungsmaximum von
\begin{equation*}
  d = \SI{7.15(7)}{cm}
\end{equation*}
Die Gitterkonstante beträgt
\begin{equation*}
  d_\symup{g} = 1,25 \cdot 10^{-5} \, \symup{m}
\end{equation*}
Aus der Beugungsbedingung
\begin{equation*}
  d_\symup{g} \cdot \sin(\theta) = d_\symup{g} \cdot \frac{d}{L} = n \cdot \lambda
\end{equation*}
ergibt sich mit den obigen Werten und $n = 1$ (da erstes Beugungsmaximum) eine
Wellenlänge von
\begin{equation*}
  \lambda = \SI{622(6)}{\nano\meter}
\end{equation*}

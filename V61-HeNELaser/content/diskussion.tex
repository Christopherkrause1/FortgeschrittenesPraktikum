\section{Diskussion}
\label{sec:Diskussion}
Die gemessenen Stromstärken weichen stark von den theoretischen Kurven der Stabilitätsbedingung ab. Dies kann auf
mehrere Unsicherheiten zurückgeführt werden. Die Angabe des Stromstärkemessgeräts schwankt deutlich, zum Beispeil durch äußere Lichteinflüsse, wodurch
die Ermittlung eines genauen Wertes schwierig ist. Nach jeder Messung müssen die Resonatorspiegel wieder
möglichst optimal justiert werden, wobei eine optimale Justage oft nur schwierig erkannt werden kann.


Bei der Vermessung der TEM-Moden sowie auch bei der Überprüfung der Polarisation
folgen die gemessenen Werte deutlich der theoretisch angenommenen Verteilung. Somit
können auch die Funktionen entsprechend gut an die Messwerte angepasst werden.
Leichte Abweichungen sind jedoch trotzdem zu erkennen. Grund dafür werden systematische
Fehler sein. Zum einen ist das Ablesen der gemessenen Stromstärke vom Messgerät nicht
einwandfrei möglich. Der angezeigte Wert schwankt während des Messvorgangs deutlich,
wodurch sich das Festlegen eines expliziten Wertes als schwierig erweist. Begünstigt wird
dieses Problem dann wiederum auch dadurch, dass die Diode, mit der die Intensität des Lasers
vermessen wird, nicht nur unter dem Einfluss des einstrahlenden Laserlichts steht, sondern
auch andere Lichteinflüsse im Versuchsraum auf diese einwirken. Des Weiteren ist auch
das Nachjustieren der Resonatorspiegel nicht immer einwandfrei möglich. Dies hat ebenfalls
eine Abweichung der gemessenen Verteilung von der theoretischen zur Folge.

Die berechnete Wellenlänge liegt wie erwartet im Bereich des roten Lichts und weicht
nur sehr leicht vom theoretisch erwarteten Wert ab. Dabei ist eine Fehlerquelle,
dass das Vermessen der Abstände mit dem gegebenen Maßband nicht einwandfrei
möglich ist und es somit zu Abweichungen bei eben diesen Werten kommt, die zur Berechnung
der Wellenlänge verwendet werden.

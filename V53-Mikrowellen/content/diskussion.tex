\section{Diskussion}
\label{sec:Diskussion}
Die Amplituden der Moden stimmen nicht mit denen des Oszilloskop überein, da die
Amplituden mit steigender Spannung größer werden sollten. Dies ist bei der
dritten Mode nicht der Fall. Dies lässt sich durch einen Fehler bei der Messung erklären.

Die berechnete Frequenz der Mikrowelle liegt dicht bei den durch den Frequenzmesser
bestimmten Frequenzen. Die Bestimmung der Wellenlänge durch den Abstand der Minima
ist somit sinnvoll.

Die berechnete Dämpfungskurve weist einen ähnlichen Verlauf wie die der Eichkurve auf.
Deutlich wird jedoch eine Verschiebung der beiden Kurven zueinander.
%Als merkliche 
%Fehlerquelle kann dabei das SWR-Meter genannt werden, da dessen Angaben sehr 
%schwanken.

\section{Diskussion}
\label{sec:Diskussion}
Die Amplituden der Moden stimmen nicht mit denen des Oszilloskop überein, da die
Amplituden mit steigender Spannung größer werden sollten. Dies ist bei der
dritten Mode nicht der Fall. Dies lässt sich durch einen Fehler bei der Messung erklären.

Die berechnete Frequenz der Mikrowelle liegt dicht bei den durch den Frequenzmesser
bestimmten Frequenzen. Die Bestimmung der Wellenlänge durch den Abstand der Minima
ist somit sinnvoll.

Die berechnete Dämpfungskurve weist einen ähnlichen Verlauf wie die der Eichkurve auf.
Deutlich wird jedoch eine Verschiebung der beiden Kurven zueinander.
%Als merkliche
%Fehlerquelle kann dabei das SWR-Meter genannt werden, da dessen Angaben sehr
%schwanken.
Bei der Bestimmung des Welligkeitsverhältnisses liegen die gemessenen und berechneten
Werte in einem durchaus realistischen Bereich ($10^0 - 10^1$). Anzumerken ist jedoch,
dass sich bei der 3dB-Methode und der Abschwächer-Methode sehr verschiedene Werte
ergeben. Die Einstellungen des Gleitschraubentransformators waren jedoch bei beiden
Messungen die selben, weshalb sich theoretisch gleiche Werte für das Welligkeitsverhältnis
ergeben sollten. Dies ist jedoch offensichtlich nicht der Fall. Grund dafür wäre ggf.
eine ungenaue Messung, welche sich aufgrund des erschwerten Ablesens der
Messwerte vom SWR-Meter ergeben könnte. Dieses ist sehr empfindlich und es kommt
daher zu starken Schwankungen des Zeigers auch während des Messvorgangs.
Dementsprechend ist auch die tatsächliche Übereinstimmung einer der Werte mit
dem tatsächlichen Wert des Welligkeitsverhältnisses fraglich. Das angesprochene
Problem wirkt sich entsprechend auch auf die Messerte der direkten Methode aus.

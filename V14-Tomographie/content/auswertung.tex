\section{Auswertung}
\label{sec:Auswertung}

\subsection{Spektrum des \textsuperscript{137}Cs-Strahlers}

Das aufgenommene Spektrum des \textsuperscript{137}Cs-Strahlers ist in Abbildung
\ref{fig:verlauf} dargestellt.

\begin{figure}[H]
  \centering
  \includegraphics{verlauf.pdf}
  \caption{Aufgenommenes Spektrum des \textsuperscript{137}Cs-Strahlers.}
  \label{fig:verlauf}
\end{figure}

Eindeutig zu erkennen ist das Compton-Kontinuum im Bereich kleiner Channel (Energien).
Im Bereich um Channel $90$ ist dann die Compton-Kante zu finden. In diesem Bereich
fällt die Anzahl an Counts noch einmal eindeutig ab. Der eigentliche Peak der $\gamma$-Strahlung der Caesium-Quelle
ist dann in einem Bereich um Channel 127 zu erkennen. Dieser hat eine gewisse Breite,
weshalb im Folgenden die Zählraten durch Integration über eben diesen Bereich im
entsprechenden Spektrum ermittelt werden.

\subsection{Abschwächung des Aluminiumwürfels}
Die gemessenen Counts $N$ des ersten Würfels, sowie die daraus resultierenden Zählraten  $I_{0,\mathrm{i}}$ bei einer Messung
über $\SI{20}{\second}$ sind in Tabelle \ref{tab:w1} dargestellt.

\begin{table}[H]
  \centering
  \caption{Zählrate in Abhängigkeit der Projektion bei einer Messdauer von $\SI{100}{\second}$ }
  \label{tab:w1}
  \begin{tabular}{c c c}
    \toprule
    Projektion & $N$ & $I_{0,\mathrm{i}} / \frac{1}{\symup{s}}$  \\
    \midrule
        $I_1$    & $\SI{16161(152)}{}$ & $\SI{161.61(152)}{}$    \\
        $I_2$    & $\SI{16165(151)}{}$ & $\SI{161.65(151)}{}$    \\
        $I_3$    & $\SI{16218(151)}{}$ & $\SI{162.18(151)}{}$    \\
        $I_4$    & $\SI{15596(151)}{}$ & $\SI{155.96(151)}{}$    \\
        $I_5$    & $\SI{15662(151)}{}$ & $\SI{156.62(151)}{}$    \\
        $I_6$    & $\SI{15899(152)}{}$ & $\SI{158.99(152)}{}$    \\
        $I_7$    & $\SI{15997(153)}{}$ & $\SI{159.97(153)}{}$    \\
        $I_8$    & $\SI{16187(152)}{}$ & $\SI{161.87(152)}{}$    \\
        $I_9$    & $\SI{16213(153)}{}$ & $\SI{162.13(153)}{}$    \\
        $I_{10}$ & $\SI{15509(151)}{}$ & $\SI{155.09(151)}{}$   \\
        $I_{11}$ & $\SI{15725(151)}{}$ & $\SI{157.25(151)}{}$    \\
        $I_{12}$ & $\SI{15605(152)}{}$ & $\SI{156.05(152)}{}$    \\
    \bottomrule
  \end{tabular}
\end{table}

Die $I_{0,\mathrm{i}}$ sind für die weiteren Berechnungen die Ausgangszählraten,
da alle anderen Würfel von einem Aluminiumwürfel ummantelt sind.


\subsection{Abschwächungskoeffizienten des zweiten Würfels}

Die vier gemessenen Zählraten $N_\symup{i}$ sind mit ihrer zugehörigen Projektion in Tabelle
\ref{tab:w2} aufgeführt. Die Zuordnung der Projektionen entspricht derjenigen aus dem
vorherigen Abschnitt. Die Messdauer beträgt $\SI{300}{\second}$.

\begin{table}[H]
  \centering
  \caption{Zählraten des zweiten Würfels in Abhängigkeit der Projektion bei einer Messdauer von $\SI{300}{\second}$ }
  \label{tab:w2}
  \begin{tabular}{c c}
    \toprule
    Projektion & $N_{\mathrm{i}} / \frac{1}{\symup{s}}$   \\
    \midrule
        $I_1$    & $\SI{25.56(36)}{}$ \\
        $I_2$    & $\SI{24.04(36)}{}$ \\
        $I_5$    & $\SI{17.13(31)}{}$ \\
        $I_6$    & $\SI{18.81(32)}{}$ \\
    \bottomrule
  \end{tabular}
\end{table}

Daraus lassen sich nach Gleichung \ref{eqn:mu} die Absorptionskoeffizienten bestimmen.
Die Kantenlänge des inneren Würfels beträgt $\SI{3}{\centi\meter}$.
Die Ergebnisse sind in Tabelle \ref{tab:mu2} aufgeführt.

\begin{table}[H]
  \centering
  \caption{Berechnete Absorptionskoeffizienten des zweiten Würfels in Abhängigkeit der Projektion}
  \label{tab:mu2}
  \begin{tabular}{c c}
    \toprule
    Projektion & $\mu_{\mathrm{i}} / \frac{1}{\symup{s}}$   \\
    \midrule
        $I_1$    & $\SI{0.615(6)}{}$ \\
        $I_2$    & $\SI{0.635(6)}{}$ \\
        $I_5$    & $\SI{0.522(5)}{}$ \\
        $I_6$    & $\SI{0.755(7)}{}$ \\
    \bottomrule
  \end{tabular}
\end{table}

Die Fehler werden hier nach der Gauß'schen Fehlerfortpflanzung bestimmt:
\begin{equation*}
  \sigma_{\mu_{\mathrm{i}}} = \sqrt{\left(\frac{1}{d\cdot I_{0,\mathrm{i}}} \cdot \sigma_{I_{0,\mathrm{i}}} \right)^2 + \left( \frac{1}{d\cdot I_{0,\mathrm{i}} \cdot N_{\mathrm{i}}}
  \cdot \sigma_{N_{\mathrm{i}}} \right)^2}
\end{equation*}

Es ergibt sich ein Mittelwert von:
\begin{equation*}
  \bar\mu_2 = \SI{0.632(3)}{1\per\centi\meter}
\end{equation*}


\subsection{Abschwächungskoeffizienten des dritten Würfels}

Die vier gemessenen Zählraten $N_\symup{i}$ sind mit ihrer zugehörigen Projektion in Tabelle
\ref{tab:w3} aufgeführt. Die Zuordnung der Projektionen entspricht derjenigen aus dem
vorherigen Abschnitt. Die Messdauer beträgt $\SI{300}{\second}$.

\begin{table}[H]
  \centering
  \caption{Zählraten des dritten Würfels in Abhängigkeit der Projektion bei einer Messdauer von $\SI{300}{\second}$ }
  \label{tab:w3}
  \begin{tabular}{c c}
    \toprule
    Projektion & $N_{\mathrm{i}} / \frac{1}{\symup{s}}$   \\
    \midrule
        $I_1$    & $\SI{109.66(75)}{}$ \\
        $I_2$    & $\SI{107.91(75)}{}$ \\
        $I_5$    & $\SI{102.99(73)}{}$ \\
        $I_6$    & $\SI{97.85(71)}{}$ \\
    \bottomrule
  \end{tabular}
\end{table}

Daraus lassen sich wieder nach Gleichung \ref{eqn:mu} die Absorptionskoeffizienten bestimmen.
Die Kantenlänge des inneren Würfels beträgt auch hier $\SI{3}{\centi\meter}$.
Die Ergebnisse sind in Tabelle \ref{tab:mu3} aufgeführt.

\begin{table}[H]
  \centering
  \caption{Berechnete Absorptionskoeffizienten des dritten Würfels in Abhängigkeit der Projektion}
  \label{tab:mu3}
  \begin{tabular}{c c}
    \toprule
    Projektion & $\mu_{\mathrm{i}} / \frac{1}{\symup{s}}$   \\
    \midrule
        $I_1$    & $\SI{0.129(4)}{}$ \\
        $I_2$    & $\SI{0.135(4)}{}$ \\
        $I_5$    & $\SI{0.099(3)}{}$ \\
        $I_6$    & $\SI{0.172(4)}{}$ \\
    \bottomrule
  \end{tabular}
\end{table}

Die Fehler werden auch hier nach der Gauß'schen Fehlerfortpflanzung bestimmt.

Es ergibt sich ein Mittelwert von:
\begin{equation*}
  \bar\mu_3 = \SI{0.134(2)}{1\per\centi\meter}
\end{equation*}


\subsection{Abschwächungskoeffizienten des vierten Würfels}
In Tabelle \ref{tab:mu} werden die gemessenen Zählraten und
die daraus berechneten Abschwächungskoeffizienten dargestellt.

\begin{table}[H]
  \centering
  \caption{Zählrate in Abhängigkeit der Projektion bei einer Messdauer von $\SI{300}{\second}$}
  \label{tab:mu}
  \begin{tabular}{c c c}
    \toprule
    Projektion & $N$ & Fehler   \\
    \midrule
        $I_1$    & 16161 & 152    \\
        $I_2$    & 16165 & 151    \\
        $I_3$    & 16218 & 151    \\
        $I_4$    & 15596 & 151    \\
        $I_5$    & 15662 & 151    \\
        $I_6$    & 15899 & 152    \\
        $I_7$    & 15997 & 153    \\
        $I_8$    & 16187 & 152    \\
        $I_9$    & 16213 & 153    \\
        $I_{10}$ & 15509 & 151   \\
        $I_{11}$ & 15725 & 151    \\
        $I_{12}$ & 15605 & 152    \\
    \bottomrule
  \end{tabular}
\end{table}

Für die Matrix $A$ der zwölf Projektionen gilt hierbei:
$$ A =
\left( \begin{matrix}
       1       & 1 &       1 &       0 &        0 &       0 &       0 &       0 &        0      \\
       0       & 0 &       0 &       1 &        1 &       1 &       0 &       0 &        0      \\
       0       & 0 &       0 &       0 &        0 &       0 &       1 &       1 &        1      \\
       0       & 0 &       0 &       0 &        0 &       \sqrt{2}& 0 &       \sqrt{2}&  0      \\
       0       & 0 &       \sqrt{2}& 0 &        \sqrt{2}& 0 &      \sqrt{2}&  0 &       0      \\
       0       & \sqrt{2}& 0 &      \sqrt{2}&   0 &       0 &       0 &       0 &       0      \\
       1       & 0 &       0 &       1 &        0 &       0 &       0 &       1 &       0      \\
       0       & 1 &       0 &       0 &        1 &       0 &       0 &       1 &       0      \\
       0       & 0 &       1 &       0 &        0 &       1 &       0 &       0 &       1      \\
       0       & \sqrt{2}& 0 &       0 &        0 &       \sqrt{2}& 0 &       0 &       0      \\
       \sqrt{2}& 0 &       0 &       0 &        \sqrt{2}& 0 &       0 &       0 &       \sqrt{2}      \\
       0       & 0 &       0 &       \sqrt{2}&  0 &       0 &       0 &       \sqrt{2} & 0
\end{matrix} \right)
$$

Die Abschwächungskoeffizienten der einzelnen Materialien werden mit Gleichung (3) bestimmt und sind in Tabelle \ref{tab:blob} dargestellt.
Die Dicke der Elementarwürfel beträgt dabei $d=\SI{1}{\centi\meter}$.
\begin{table}[H]
  \centering
  \caption{Abschwächungskoeffizienten und zugehörige Fehler des vierten Würfels}
  \label{tab:blob}
  \begin{tabular}{c c}
    \toprule
    $\mu/ \mathrm{\frac{1}{cm}}$ & Fehler   \\
    \midrule
    0.204      &  0.009 \\
    0.404      &  0.006 \\
    0.304      &  0.009 \\
    0.458      &  0.006 \\
    0.311      &  0.007 \\
    0.082      &  0.006 \\
    0.381      &  0.012 \\
    0.056      &  0.006 \\
    0.319      &  0.009 \\
    \bottomrule
  \end{tabular}
\end{table}


Die Fehler der Abschwächungskoeffizienten werden dabei mit der Gaußschen Fehlerfortpflanzung von Python berechnet.
\begin{equation}
  \sigma_f = \sqrt{
      \sum\limits_{i = 1}^N
       \left( \frac{\partial f}{\partial x_i} \sigma_i \right)^{\!\! 2}
     }
\end{equation}

Für den Fehler der einzelnen Absorptionskoeffizienten $\mu_{\mathrm{i}}= \symbf{\tilde{A}} \cdot I_{\mathrm{i}}$ gilt:
\begin{align*}
  \sigma_{\mu_i} = \left(\sqrt{\frac{\symbf{\tilde{A}}^2 \sigma_{N}^2}{N^2} + \frac{\symbf{\tilde{A}}^{2} \sigma_{I_{0}}^2}{I_{0}^2}}\right)_i \\
  \text{mit}\:\: \symbf{\tilde{A}} = \left(\symbf{A^T} \cdot \symbf{A}\right)^{-1} \symbf{A^T}
\end{align*}

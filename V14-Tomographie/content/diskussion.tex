\section{Diskussion}
\label{sec:Diskussion}

Das aufgenommene Spektrum des \textsuperscript{137}Cs-Strahlers entspricht den
theoretischen Erwartungen. Bei der Wechselwirkung der $\gamma$-Strahlung mit den
Atomen des Szintillatordetektors kann es durch den Compton-Effekt dazu kommen, dass
nicht die gesamte Energie der Strahlung im Detektor deponiert wird. Folglich werden
auch in einem Bereich kleinerer Energien Ereignisse gemessen. Im Spektrum zu erkennen
als das Compton-Kontinuum im Bereich kleinerer Channel. Darauf folgt der Bereich,
in dem kein Energieübertrag durch den Compton-Effekt mehr möglich ist, wodurch die
Zählrate sinken sollte. Auch dieser Abfall ist im aufgenommenen Spektrum zu erkennen.
Schließlich sollte sich ein Peak ergeben, welcher durch die vollständige Energieabgabe
der $\gamma$-Strahlung zustande kommt (vollständige Energieabgabe druch Photoeffekt).
Dieser wird ebenfalls im Spektrum nachgewiesen.

Im Folgenden werden die berechneten Werte der Absoprtionskoeffizienten mit den
entsprechenden Literaturwerten $\mu_{\mathrm{lit}}$ verglichen. Diese sind in Tabelle \ref{tab:literatur}
aufgeführt.

\begin{table}[H]
  \centering
  \caption{Literaturwerte der Absoprtionskoeffizienten für die verschiedenen Materialien \cite{sample2}.}
  \label{tab:literatur}
  \begin{tabular}{c c}
    \toprule
    Material &  $\mu_{\mathrm{lit}} / \frac{1}{\mathrm{cm}}$  \\
    \midrule
        Aluminium    & $\SI{0.211}{}$ \\
        Blei    & $\SI{1.415}{}$ \\
        Eisen & $\SI{0.609}{}$ \\
        Messing    & $\SI{0.631}{}$ \\
        Delrin & $\SI{0.122}{}$ \\
    \bottomrule
  \end{tabular}
\end{table}

Der Absoprtionskoeffizient des zweiten Würfels $\bar\mu_2 = \SI{0.632(3)}{1\per\centi\meter}$
stimmt mit einer relativen Abweichung von lediglich $0,1$\% sehr gut mit dem Literaturwert
von Messing überein. Somit liegt nahe, dass Würfel 2 aus Messing bestehen muss.

Der Absorptionskoeffizient des dritten Würfels $\bar\mu_3 = \SI{0.134(2)}{1\per\centi\meter}$
passt mit einer relativen Abweichung von $9,8$\% am besten zum Literaturwert von
Delrin. Daher sollte dieser Würfel also vollständig aus Delrin bestehen.

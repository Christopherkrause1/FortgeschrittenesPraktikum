\section{Diskussion}
\label{sec:Diskussion}
Das Material des zweiten Würfels wird mit Messing identifiziert. Der berechnete
Absorptionskoeffizient $\bar\mu_2 = \SI{0.632(3)}{1\per\centi\meter}$ weicht um
lediglich $0,1$\% vom Literaturwert $\mu_{2, \mathrm{ lit}} = \SI{0.631}{1\per\centi\meter}$
ab.

Das Material des dritten Würfels wird mit Delrin identifiziert. Der berechnete
Absorptionskoeffizient $\bar\mu_3 = \SI{0.134(2)}{1\per\centi\meter}$ weicht um
$9,8$\% vom Literaturwert $\mu_{3, \mathrm{ lit}} = \SI{0.122}{1\per\centi\meter}$
ab.

Trotz der nicht allzu großen Abweichungen sind die gemachten Identifikationen
kritisch zu betrachten. Die Ausrichtung der einzelnen Würfel zum Strahlengang
ist nicht mit größter Genauigkeit möglich. Dadurch entstehen Abweichungen der
berechneten Absorptionskoeffizienten zum tatsächlichen Wert.
Zusätzlich beziehen sich die Literaturwerte der Abschwächungskoeffizienten auf
eine Photonenenergie von $\SI{600}{\kilo\eV}$, wobei der eigentliche Peak
bei $\SI{662}{\kilo\eV}$ ist. Auch dadurch ist die Zuordnung im Prinzip nicht
eindeutig möglich.

Die Abschwächungskoeffizienten des vierten Würfels und der dazu am naheliegendste Stoff wird in
Tabelle \ref{tab:abw} dargestellt.

\begin{table}[H]
  \centering
  \caption{Abschwächungskoeffizienten und zugehörige Stoffe des vierten Würfels}
  \label{tab:abw}
  \begin{tabular}{c c c}
    \toprule
    $\mu/ \mathrm{\frac{1}{cm}}$ & Stoff & Abweichung   \\
    \midrule
    0,204      &  Al     &   0,03    \\
    0,404      &  Fe, Me &   0,34, 0,36    \\
    0,304      &  Al     &  -0,44    \\
    0,458      &  Fe, Me &   0,25, 0,27    \\
    0,311      &  Al     &   0,47    \\
    0,082      &  De     &   0,33    \\
    0,381      &  Al     &  -0,81    \\
    0,056      &  De     &   0,54   \\
    0,319      &  Al     &  -0,51  \\
    \bottomrule
  \end{tabular}
\end{table}

Die Abweichungen sind bei fast allen Stoffen mit über $30\%$ so groß, dass
eine genaue Zuordnung in mehreren Fällen nicht eindeutig ist. Gerade bei Messing und
Eisen, welche einen ähnlichen Abschwächungskoeffizient besitzen ist eine genaue
Aussage nicht möglich. Für Werte im Bereich von $\mu = 0.35$ ist eine Aussage, ob
dieser zu dem nächsten kleineren oder nächsten größeren Abschwächungskoeffizient
gehört. Die Abweichungen von den theoretischen Werten können ebenfalls durch
die oben bereits genannten Fehlerquellen erklärt werden.

\section{Diskussion}
\label{sec:Diskussion}

Im Folgenden werden die berechneten Werte der Absorptionskoeffizienten mit den
entsprechenden Literaturwerten $\mu_{\mathrm{lit}}$ verglichen. Diese sind in Tabelle \ref{tab:literatur}
aufgeführt.

\begin{table}[H]
  \centering
  \caption{Literaturwerte der Absorptionskoeffizienten für die verschiedenen Materialien bei einer
  Photonenenergie von $\SI{600}{\kilo\eV}$ \cite{sample2}.}
  \label{tab:literatur}
  \begin{tabular}{c c}
    \toprule
    Material &  $\mu_{\mathrm{lit}} / \frac{1}{\mathrm{cm}}$  \\
    \midrule
        Aluminium    & $\SI{0.211}{}$ \\
        Blei    & $\SI{1.415}{}$ \\
        Eisen & $\SI{0.609}{}$ \\
        Messing    & $\SI{0.631}{}$ \\
        Delrin & $\SI{0.122}{}$ \\
    \bottomrule
  \end{tabular}
\end{table}

Der Absorptionskoeffizient des zweiten Würfels $\bar\mu_2 = \SI{0.632(3)}{1\per\centi\meter}$
stimmt mit einer relativen Abweichung von lediglich $0,1$\% sehr gut mit dem Literaturwert
von Messing überein. Somit liegt nahe, dass Würfel 2 aus Messing bestehen muss.

Der Absorptionskoeffizient des dritten Würfels $\bar\mu_3 = \SI{0.134(2)}{1\per\centi\meter}$
passt mit einer relativen Abweichung von $9,8$\% am besten zum Literaturwert von
Delrin. Daher sollte dieser Würfel also vollständig aus Delrin bestehen.



Die Abschwächungskoeffizienten und der dazu am naheliegendste Stoff wird in
Tabelle \ref{tab:abw} dargestellt.

\begin{table}[H]
  \centering
  \caption{Abschwächungskoeffizienten und zugehörige Stoffe des vierten Würfels}
  \label{tab:abw}
  \begin{tabular}{c c c}
    \toprule
    $\mu/ \mathrm{\frac{1}{cm}}$ & Stoff & Abweichung   \\
    \midrule
    0,204      &  Al     &   0,03    \\
    0,404      &  Fe, Me &   0,34, 0,36    \\
    0,304      &  Al     &  -0,44    \\
    0,458      &  Fe, Me &   0,25, 0,27    \\
    0,311      &  Al     &   0,47    \\
    0,082      &  De     &   0,33    \\
    0,381      &  Al     &  -0,81    \\
    0,056      &  De     &   0,54   \\
    0,319      &  Al     &  -0,51  \\
    \bottomrule
  \end{tabular}
\end{table}

Die Abweichungen sind bei fast allen Stoffen mit über $30\%$ so groß, dass
eine genaue Zuordnung in mehreren Fällen nicht eindeutig ist. Gerade bei Messing und
Eisen, welche einen ähnlichen Abschwächungskoeffizient besitzen ist eine genaue
Aussage nicht möglich. Für Werte im Bereich von $\mu = 0.35$ ist eine Aussage, ob
dieser zu dem nächsten kleineren oder nächsten größeren Abschwächungskoeffizient
gehört. Die Abweichungen von den theoretischen Werten können durch ungenaue Ausrichtungen des
Aluminiumwürfels und des vierten Würfels zum Strahlengang erklärt werden.
Zusätzlich beziehen sich die Literaturwerte der Abschwächungskoeffizienten auf
eine Photonenenergie von $\SI{600}{\kilo\eV}$, wobei der eigentliche Peak
bei $\SI{662}{\kilo\eV}$ ist.

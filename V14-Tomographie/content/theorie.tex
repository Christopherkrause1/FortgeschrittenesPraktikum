\section{Theorie}
\label{sec:Theorie}

Die Tomographie ist ein bildgebendes Verfahren um ein räumliches Bild von inneren Strukturen
eines Körpers zu erlangen. Dabei wird der Körper aus verschiedenen Positionen mit
Gamma-Strahlen bestrahlt. Besteht der Körper aus unterschiedlichen Elementen und Verbindungen,
so werden die Strahlen unterschiedlich stark absorbiert. Für die Ausgangsintensität $I$
der Strahlung gilt:
\begin{align}
  &N=I_{\symup{0}} \exp{\left(-\Sigma \mu_{\symup{i}} d_{\symup{i}}\right)} \\
  \Rightarrow &\Sigma \mu_{\symup{i}} d_{\symup{i}} = \ln{\frac{I_{\symup{0}}}{N}}
  \label{eqn:mu}
\end{align}
Mit der Anfangsintensität $I_{\symup{0}}$, den Schichtdicken $d_{\symup{i}}$
und den Absorptionskoeffizienten $\mu_{\symup{i}}$.
Gleichung (2) gilt für jede Projektion, wodurch ein Gleichungssystem entsteht. Um eine höhere
Messgenauigkeit zu erhalten, sollten mehr Projektion gemacht werden, als es
Materialien im Würfel gibt.

Die Absorptionskoeffizienten $\mu_{\symup{j}}$ werden bestimmt durch:
\begin{align}
  \mu_{\symup{j}} = \left[\left(\symbf{A^T} \cdot \symbf{A})\right)^{-1} \symbf{A^T}\right]_{\symup{j}} \cdot \ln{\frac{I_{\symup{0}}}{N_{\symup{j}}}}
\end{align}
Die Matrix $\symbf{A}$ hängt hierbei von der Würfelgeometrie und den Projektionen ab.

Die Abschwächung von Gamma-Strahlen kann durch 3 verschiedene Prozesse eintreten.
$\symbf{1.}$ Photoeffekt: Die Photonen regen ein Elektron an und geben ihre vollständige Energie an diese ab.
$\symbf{2.}$ Comptoneffekt: Die Photonen streuen an den Elektronen, wodurch ein Teil ihrer Energie
an die Elektronen abgegeben wird.
$\symbf{3.}$ Paarbildung: In der Nähe der Atomkerne können die Photonen sich in ein Elektron und ein Positron
umwandeln. Dies geht jedoch nur, wenn die Photonen mindestens eine Ruhenergie von 2 Elektronen
haben.
